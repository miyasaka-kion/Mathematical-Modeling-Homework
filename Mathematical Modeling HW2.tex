\documentclass[11pt]{article}
\usepackage{amsfonts}
\usepackage{amsmath}
\usepackage[normalem]{ulem}
\usepackage[a4paper, total={6in, 8in}]{geometry}

\title{\textbf{Mathematical Modeling HW2}}
\author{\textsc{Miyasaka Kion}}
\date{\today}

\begin{document}	
\maketitle



\section{Whales}
1. Ecologists use the following model to represent the growth process of two competing species, $x$ and $y$ :

$$
\begin{aligned}
&\frac{d x}{d t}=r_{1} x\left(1-\frac{x}{K_{1}}\right)-\alpha_{1} x y \\
&\frac{d y}{d t}=r_{2} y\left(1-\frac{y}{K_{2}}\right)-\alpha_{2} x y
\end{aligned}
$$

The variables $x$ and $y$ represent the number in each population; the parameters $r_{i}$ represent the intrinsic growth rates of each species; $K_{i}$ represents the maximum sustainable population in the absence of competition; and $\alpha_{i}$ represents the effects of competition. Studies of the blue whale and fin whale populations have determined the following parameter values $(t$ in years): 
\begin{center}
\begin{tabular}{ccc}
\hline & Blue & Fin \\
\hline$r$ & $0.05$ & $0.08$ \\
$K$ & 150,000 & 400,000 \\
$\alpha$ & $10^{-8}$ & $10^{-8}$ \\
\hline
\end{tabular}\\
\end{center}
\subsection{Question}
(a) Determine the population levels $x$ and $y$ that maximize the number of new whales born each year. Use the five-step method, and model as an unconstrained optimization problem.\\
\paragraph{Step 1.} Ask the question\\

$$
\begin{aligned}
\text {Variables:}\\
x &= \text {population of blue whale (whale)}\\
y &= \text {population of fin whale (whale)}\\
B &= \text {growth rate of blue whale (whale/ year)}\\
F &= \text {growth rate of fin whale (whale/ year)}\\
t &= \text {current year (year)}\\
T &= \text {total growth rate (whale/year)}\\
\text{Constants:}\\
r_1 &= \text {intrinsic growth rate of blue whale}\\
r_2 &= \text {intrinsic growth rate of fin whale}\\
K_1 &= \text {maximum sustainable population of blue whale, in absence of competition }\\
K_2 &= \text {maximum sustainable population of fin whale, in absence of competition }\\
\alpha _1&= \text{effect of competition, w.r.t blue whale}\\
\alpha _2&= \text{effect of competition, w.r.t fin whale}\\
\\
\text{Assumptions:}\\
B &= r_{1} x\left(1-\frac{x}{K_{1}}\right)-\alpha_{1} x y \\
F &=r_{2} y\left(1-\frac{y}{K_{2}}\right)-\alpha_{2} x y\\
T &= B+F\\
x& \geq 0 \\
y&\geq 0 \\
\text{Objective:}&\quad  \text{maximize T}\\
\end{aligned}
$$

\paragraph{Step 2.} Select the modeling approach.

Model this problem as an unconstrained optimize problem.
\paragraph{Step 3. }Formulate the model.

 Substitude constants into the equations of $B$ and $F$
$$
\begin{align}
&B = 0.05 x\left(1-\frac{x}{150000}\right)-10^{-8} x y \\
&F = 0.08 y\left(1-\frac{y}{400000}\right)-10^{-8} x y\\
\end{align}
$$


let $f(x,y) = T = B+F$

$$
\begin{align}
f(x,y) &=  0.05 x\left(1-\frac{x}{150000}\right)-10^{-8} x y + 0.08 y\left(1-\frac{y}{400000}\right)-10^{-8} x y\\
f(x,y) &= -\frac{x y}{50000000}+0.05 \left(1-\frac{x}{150000}\right) x+0.08 \left(1-\frac{y}{400000}\right) y\end{align}
$$


\paragraph{Step 4.} Solve the model.

Take the derivative of $f$, yields 
$$
\begin{aligned}
\nabla f  = \left\{&0.05 \left( 1-\frac{x}{150000}\right) -\text{3.3333333333333335$\grave{ }$*${}^{\wedge}$-7} x-\frac{y}{50000000},\\
&-\frac{x}{50000000}+0.08 \left(1-\frac{y}{400000}\right)-\text{2.0000000000000002$\grave{ }$*${}^{\wedge}$-7} y\right\}
\end{aligned}
$$

letting $\nabla f = 0$, we have 

$$
\begin{cases}
	0.05 \left(1-\frac{x}{150000}\right)-\text{3.3333333333333335$\grave{ }$*${}^{\wedge}$-7} x-\frac{y}{50000000} &= 0\\ \\
	-\frac{x}{50000000}+0.08 \left(1-\frac{y}{400000}\right)-\text{2.0000000000000002$\grave{ }$*${}^{\wedge}$-7} y &= 0
\end{cases}
$$

By solving the equation of $x$ and $y$ above, we get

$$
x = 69103.7, y = 196545,f(x,y)= 9589.38
$$

since $x \in \mathbb Z$, pick $x= 69104 , y = 196545$, and $f=9589.38 \approx 9589$.

\paragraph{Step 5.} Answer the question.

By the calculation above, the maximum number of new born  whale is $9589$, when the population of blue whale is $69104$ and fin whale is  $196545$.

\\


\subsection{Question}
(b) Examine the sensitivity of the optimal population levels to the intrinsic growth rates $r_{1}$ and $r_{2}$.

First we examine the sensitivity of $r_1$.

$$
f(x,y,r_1) =  r_1 x\left(1-\frac{x}{150000}\right)-10^{-8} x y + 0.08 y\left(1-\frac{y}{400000}\right)-10^{-8} x y
$$
let $\nabla f = 0$, we have
$$
\begin{align}
	x\to \frac{300.\, -75000. \text{r1}}{0.000075\, -1. \text{r1}},y\to -\frac{196250. \text{r1}}{0.000075\, -1. \text{r1}}
\end{align}
\\
$$
$$
\frac{\mathrm d x}{\mathrm d r_1}  =-\frac{294.375}{(0.000075\, -1. \text{r1})^2} = -118104
$$


when $r_1 = 0.05$.

$$
S(x,r_1) = \frac{\mathrm d x}{\mathrm d r_1} \cdot \frac{r_1}{x} = 0.0854542
$$


Similarly, 


$$
S(y,r_1) = \frac{\mathrm d y}{\mathrm d r_1} \cdot \frac{r_1}{y} = -0.00150225
$$


Now compute $S(T,r_1)$.
$$
\begin{align*}
S(T,r_1) &= 
\frac{\mathrm d f}{\mathrm d r_1}\cdot \frac{r_1}{T}
 \\
&=\left( \frac{\partial f}{\partial x}\cdot \frac{\mathrm d x}{\mathrm d r_1}  + \frac{\partial f}{\partial y}\cdot \frac{\mathrm d y}{\mathrm d r_1} + \frac{\partial f}{\partial r_1}\cdot 1 \right)\cdot \frac{r_1}{T}
\\
&=\left(  \frac{\partial f}{\partial r_1} \right)\cdot \frac{r_1}{T}
\\
&= 0.19432
  \end{align*}
$$

Then we examine the sensitivity of $r_2$. The objective function is
$$
f(x,y,r_2) =  0.05 x\left(1-\frac{x}{150000}\right)-10^{-8} x y + r_2 y\left(1-\frac{y}{400000}\right)-10^{-8} x y
$$


with the similar method describe above,
$$
\nabla f = 0 \Rightarrow x\to -\frac{69000. \text{r2}}{0.00012\, -1. \text{r2}},y\to \frac{300.\, -200000. \text{r2}}{0.00012\, -1. \text{r2}} 
\\
$$


Hence
$$
\begin{align}
	S(x,r_2) & = -0.00150225\\
	S(y,r_2) & = 0.017606\\
	S(T,r_2) &  = 0.834007
\end{align}
$$

\\
\sout{
(c) Examine the sensitivity of the optimal population levels to the environmental carrying capacities $K_{1}$ and $K_{2}$.
(d) Assuming that $\alpha_{1}=\alpha_{2}=\alpha$, is it ever optimal for one species to become extinct?
}

\section{PC}
6. A manufacturer of personal computers currently sells 10,000 units per month of a basic model. The cost of manufacture is $\$ 700 /$ unit, and the wholesale price is $\$ 950$. During the last quarter the manufacturer lowered the price $\$ 100$ in a few test markets, and the result was a $50 \%$ increase in sales. The company has been advertising its product nationwide at a cost of $\$ 50,000$ per month. The advertising agency claims that increasing the advertising budget by $\$ 10,000 /$ month would result in a sales increase of 200 units/month. Management has agreed to consider an increase in the advertising budget to no more than $\$ 100,000 /$ month.
\subsection{Question}
(a) Determine the price and the advertising budget that will maximize profit. Use the five-step method. Model as a constrained optimization problem, and solve using the method of Lagrange multipliers.
 
\paragraph{Step 1.}
$$
\begin{aligned}
	\text{Variables:}\\
	p &= \text{price of a unit (\$)}\\
	a &= \text{advertise investigation increase(\$/mo)}\\
	d &= \text{price drop per PC(\$/mo)}\\
	t &= \text{total amount of PCs sells(/mo)}\\
	A &= \text{total advertising cost(\$/mo)}\\
	R &= \text{total revenue(\$/mo)}\\
	C &= \text{total cost(\$/mo)}\\
	P &= \text{profit(\$/mo)}\\
	\text{Assumptions:}\\
	p &= 950 - d\\
	A &= 50000 + a\\
	t &= 10000 + 50d + a/50  \\
	C &= A +  t\times 700\\
	R &= tp\\
	P &= R - C\\
	t & \geq 0 \\
	50000\leq A&\leq 100000\\
	d & \geq 0\\
	p & \leq 950\\
	50000\geq a & \geq 0\\
	\text{Objective:}\\
	&\text{Maximize}  \quad  P 
\end{aligned}
$$
 \paragraph{Step 2.} Select the modeling approach.
 
 We treat this problem as a constrained optimization problem.
 
 \paragraph{Step 3.} Formulate the model.
 
$$
 \begin{aligned}
 P & = R - C \\
 & = tp  - (A + 700t)\\
 &= (10000 + 50d + a/50)(950-d) - (50000 + a) - 700(10000 + 50d + a/50)\\
 &= a \left(4-\frac{d}{50}\right)-50 \left(d^2-50 d-49000\right)
 \end{aligned}
$$
 
then the original problem converts to find the maximum value of $f(x,y)$ where
$$
f(x,y) = x \left(4-\frac{y}{50}\right)-50 \left(y^2-50 y-49000\right) , 0\leq x\leq 50000, 0\leq y \leq 250
$$

\paragraph{Step 4.} Solve the model.

Taking the derivative of $f$,
$$
\nabla f = \left\{4-\frac{y}{50},-\frac{x}{50}-50 (2 y-50)\right\}
$$

if we treat the problem as an unconstrained problem, by solving $\nabla f = 0$, we earn 
$$
{x = -875000, y = 200}
$$

which is not lying in the feasible region of the original constrained problem. Hence the feasible optimized solution lies on the boundary of the feasible region.

\subparagraph{Case 1.} Assume that the optimal solution lies on $g(x,y)  = y  = 0$ or 250, then $\nabla g = (0,1)$, let $\nabla f = \lambda \nabla g$ yields:
$$
4-\frac{y}{50} = 0 ,\quad 
-\frac{x}{50}-50 (2 y-50) = \lambda, 
$$

which is a trivial case, no solution.

\subparagraph{Case 2.} Assume that the optimal solution lies on $g(x,y)  = x  = 0$ or 50000, then $\nabla g = (1,0)$, let $\nabla f = \lambda \nabla g$ yields:
$$
4-\frac{y}{50} = \lambda ,\quad 
-\frac{x}{50}-50 (2 y-50) = 0, 
$$

When $x = 0$ we get $y = 25, \lambda =  7/2$ and $f =2481250$.

When $x = 50000$ we get $y = 15, \lambda =  37/10$ i.e. $f =2661250$.

Hence the maximum feasible solution is $x=50000,y = 15$, and the objective value is $f =2661250$.

\paragraph{Step 5.} Answer the question.

If the manufacturer sell the price of PC at $\$ 935 (=950 - 15)$ and use the maximum budget to advertising, they will get the maximum profit at $\$ 2661250$.

\subsection{Question}


(b) Determine the sensitivity of the decision variables (price and advertising) to price elasticity (the $50 \%$ number).
  
  \subsubsection{Sensitivity of price to price elasticity}
 Substitute 50 by $e$, the objective function becomes
 
 $$
f(x,y,e) = -e y^2+250 (e-40) y+x \left(4-\frac{y}{50}\right)+2450000
$$

From the information we have calculated in the last question, we have $x = 50000$. Then we can let $\grad f = \lambda \nabla g$, where $\nabla g = (1,0)$, 

$$
y= \frac{125 (e-44)}{e},\lambda = -\frac{-3 e-220}{2 e}
$$

$$
\begin{aligned}
	S(p,e )&= \frac{\mathrm d p }{\mathrm d e}\cdot \frac{e}{p} = -0.117647\\
	S(A,e )&= \frac{\mathrm d A }{\mathrm d e}\cdot \frac{e}{A} = 0\\
\end{aligned}
$$
 
which means when $e$ increase by 100\%, the price of PC decreases by 11.76\%.

\subsubsection{Sensitivity of advertising to price elasticity}
Substitute 200a by k, which is the advertising price elasticity
$$
f(x,y,k) = x \left(-\frac{k (y-250)}{10000}-1\right)-50 \left(y^2-50 y-49000\right)
$$

Solving the equation above using Lagrange multipliers. Let $\nabla f  = \lambda \nabla g$, where $\nabla g = (1,0), x = 50000$.

$$
\begin{aligned}
	S(p,k )&= \frac{\mathrm d p }{\mathrm d k}\cdot \frac{k}{p} = 0.0106952\\
	S(A,k )&= \frac{\mathrm d A }{\mathrm d k}\cdot \frac{k}{A} = 0\\
\end{aligned}
$$

Which means if the elasticity price $k$ grows by 100\%, the corresponding price of units will be affected by a growth of 1.07\%.




\sout{(c) Determine the sensitivity of the decision variables to the advertising agency's estimate of 200 new sales each time the advertising budget is increased by $\$ 10,000$ per month.}
    
\subsection{Question}
(d) What is the value of the multiplier found in part (a)? What is the real-world significance of the multiplier? How could you use this information to convince top management to lift the ceiling on advertising expenditures?

The Lagrange multiplier implicitly represents the rate of growth of the profit with respect to the investigation in advertising. Around the maximum profit point, an investigate of \$1 results in a \$3.7 profit.


\section{Lagrange multipliers method}
7. Solve the following constrained optimization problem by Lagrange multipliers method

$$
\begin{aligned}
\text { Minimize } & f\left(x_{1}, x_{2}\right)=2 x_{1}^{2}-2 x_{1} x_{2}+2 x_{2}^{2}-6 x_{1} \\
\text { Subject to } &\left\{\begin{array}{l}
g_{1}\left(x_{1}, x_{2}\right)=3 x_{1}+4 x_{2}-6 \leq 0 \\
g_{2}\left(x_{1}, x_{2}\right)=-x_{1}+4 x_{2}-2 \leq 0
\end{array}\right.
\end{aligned}
$$

Hint for 2:
Method 1. For this problem, the minimum is taken at the boundary of the feasible region. Method 2. Turn the inequality constraints into equality constraints by adding relaxation parameters $x_{3}$ and $x_{4}$. e.g. $3 x_{1}+4 x_{2}-6 \leq 0 \Leftrightarrow 3 x_{1}+4 x_{2}-6+x_{3}^{2}=0$.


\paragraph{Sol.} let $x = x_1 $ and $ y  = x_2$.

$$
\nabla f = \{4 x-2 y-6,4 y-2 x\}
$$

By solving $\nabla f = 0$ we have $x = 2 $ and $ y = 1$, which is not in the feasible region. Thus the optimal value must lies on the boundary of the feasible set.
We can apply the Lagrange multiplier method as below,

$$
\begin{aligned}
\nabla f &= \lambda \nabla g_1  + \mu \nabla g_2	\\
\text{i.e.}\quad  \{4 x-2 y-6,4 y-2 x\} & = \lambda \{3,4\} + \mu \{-1,4\}
\end{aligned}
$$

\paragraph{Case 1.}
$$
\begin{cases}
	\{-6 + 4 x - 2 y, -2 x + 4 y\}  = \lambda \{3, 4\} + \mu \{-1, 4\},\\\quad 3 x + 4 y - 6 = 0 
\end{cases}
$$

And the solution to the above equation is 
$$
\mu \to -\frac{37 \lambda }{17}-\frac{12}{17},x\to \frac{16 \lambda }{17}+\frac{30}{17},y\to \frac{3}{17}-\frac{12 \lambda }{17}
$$

substitute it into the origin function $f$,
$$
f\left(\frac{16 \lambda }{17}+\frac{30}{17},\frac{3}{17}-\frac{12 \lambda }{17}\right) = \frac{2}{289} \left(592 \lambda ^2+384 \lambda -711\right)
$$
which is a quadratic function, so $f_{min} = -\frac{198}{37}$ when $\lambda = -\frac{12}{37}$. Consequently, $x =  \frac{54}{37},y =  \frac{15}{37}$ can be check that it is in the feasible region.

\paragraph{Case 2.}
$$
\begin{cases}
	\{-6 + 4 x - 2 y, -2 x + 4 y\}  = \lambda \{3, 4\} + \mu \{-1, 4\},\\\quad -x + 4 y - 2 = 0 
\end{cases}
$$

And the solution to the above equation is 
$$
\mu \to -\frac{1}{13} (17 \lambda ),x\to \frac{16 \lambda }{13}+2,y\to \frac{4 \lambda }{13}+1
$$

substitute it into the origin function $f$,
$$
f = \frac{32 \lambda ^2}{13}-6
$$

which is a quadratic function, so $f_{min} = -6$ when $\lambda = 0$. However, $x =  2,y =1$ is not a feasible solution.

Hence, the solution to the optimize problem is
$$
f_{min} = -\frac{198}{37},\text{at }x =  \frac{54}{37},y =  \frac{15}{37}
$$



\end{document}