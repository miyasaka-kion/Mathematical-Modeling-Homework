\documentclass[11pt]{article}
\usepackage{amsfonts}
\usepackage{amsmath}
\usepackage[normalem]{ulem}
\usepackage[a4paper, total={6in, 8in}]{geometry}

\title{\textbf{Mathematical Modeling HW1}}
\author{\textsc{Miyasaka Kion}}
\date{\today}

\begin{document}	
\maketitle

\section{Automobile}
1. An automobile manufacturer makes a profit of $\$ 1,500$ on the sale of a certain model. It is estimated that for every $\$ 100$ of rebate, sales increase by $15 \%$.

\subsection{Question}
(a) What amount of rebate will maximize profit? Use the five-step method, and model as a one-variable optimization problem.
\paragraph{Step 1.} Ask the question.
$$
\begin{aligned}
\text{Variables:}\\
  r&= \text{money on rebate(\$)}\\
  P &= \text{profit(\$)}\\
  A& =\text{the amount of automobile will be sold}\\
\text{Assumptions:}\\
 0 &\leq r \leq 1500 \\
 A &\geq 0\\
\text{Objective:}\\
& \text{Maximize} \quad P 
\end{aligned}
$$

\paragraph{Step 2.} Select the modeling approach.

We model this problem as a one-variable optimization problem.
\paragraph{Step 3.} Formulate the model.
$$
P (r)= (1500 - r) \times A(1+ 0.15(r /100))
$$

\paragraph{Step 4.} Solve the model.
$$
P (r)= (1500 - r) \times A(1+ 0.15(r /100)) = A(-0.0015 r^2+1.25 r+1500)
$$
which is a quadratic function and it reaches the global maximum at $r =416.667 $, $P_{max}  = 1760.42A$ 


\paragraph{Step 5.} Answer the question.

A rebate of \$416.667 will make the profit maximum, which is \$1760.42 times the vehicles sold.



\subsection{Question}
(b) Compute the sensitivity of your answer to the $15 \%$ assumption. Consider both the amount of rebate and the resulting profit.
Substitute $15 \%$ by $s(s>0)$. Then $P(r)$ becomes

$$
P(r) = (1500-r) \left(\frac{r s}{100}+1\right)= -\frac{r^2 s}{100}+15 r s-r+1500,
$$

which is a quadratic function. Hence $P_{max}  = \frac{25 \left(225 s^2+30 s+1\right)}{s}$ at $r = \frac{50 (15 s-1)}{s}$. Denote the sensitivity of $P$ to $s$ as $S(P,s)$, and sensitivity of $r$ to $s$ as $S(r,s)$,

$$
\begin{align*}
	S(P,s) &= \frac{\textrm d P}{\textrm d s} \cdot \frac{s}{P}=\frac{\left(5625-\frac{25}{s^2}\right) s}{P} = 0.384615
	\\
	S(r,s) & = \frac{\textrm d r}{\textrm d s} \cdot \frac{s}{r} =\frac{50}{r s}  = 0.799999
\end{align*}
$$

\subsection{Question}
(c) Suppose that rebates actually generate only a $10 \%$ increase in sales per $\$ 100$. What is the effect? What if the response is somewhere between 10 and $15 \%$ per $\$ 100$ of rebate?

By the sensitivity computed in (b), if the rebate descend from 15\% to 10\%, i.e. a 1/3 reduction, the profit will decease by $S(P,s) \times 1/3 = 0.128205 \approx 12.82\%$ and the rebate value will decrease by $S(r,s) \times 1/3 = 0.266666 \approx 26.67\%$.

\sout{(d) Under what circumstances would a rebate offer cause a reduction in profit?}
\section{Pig}
2. In the pig problem, perform a sensitivity analysis based on the cost per day of keeping the pig. Consider both the effect on the best time to sell and on the resulting profit. If a new feed costing 60 cents/day would let the pig grow at a rate of $7 \mathrm{lbs} /$ day, would it be worth switching feed? What is the minimum improvement in growth rate that would make this new feed worthwhile?
\subsection{Sensitivity Analysis}
\paragraph{Sol.}
The objective function is

$$
 P =(0.65-0.01 t)(200+5 t)-c t  = t (1.25\, -1. c)-0.05 t^2+130.
$$

which is a quadratic function, and its maximum value is $5c^2-12.5 c+137.8134$ at $t = -10 ( c-1.25)$. The sensitivity of $P$ to $c$ is $S(P,c)$, and the sensitivity of $t$ to $c$ is $S(t,c)$.
$$
\begin{align*}
	S(P,c) &= \frac{\textrm d P}{\textrm d c} \cdot \frac{c}{P} = \frac{c (10. c-12.5)}{P} = -0.027027 \approx -2.7027\%
\end{align*}
$$
which means if the cost of keeping the pig per day grows 100\% will result in a 2.7027\% decrease of the profit.

\subsection{change of method}
If a new feed costing 60 cents/day would let the pig grow at a rate of $7 \mathrm{lbs} /$ day,  the objective function becomes
$$
P(t) = (0.65 - 0.01 t) (200 + 7 t) - 0.6 t = -0.07 t^2+1.95 t+130,
$$

which is a quadratic function and it reaches maximum at $t = 13.9286$, since $t\in \mathbb Z$, the maximum value is $P(14)  = 143.58$ at $t = 14$. The new profit is greater than the original one, thus it worth to switch feed. 

\subsection{Minimum in rate of growth}
Assume that the rate of growth is $g(g>0)$, then the objective function is 
$$
 P =(0.65-0.01 t)(200+g t)-0.6 t 
$$
and we hope $P_{max} \geq 133.2$.
Notice that 
$$
P =(0.65-0.01 t)(200+g t)-0.6 t = -0.01 g t^2+(0.65 g-2.6) t+130.
$$
which is a quadratic function. Hence $P_{max} = 45.5\, +10.5625 g+\frac{169.}{g}$ at $t = \frac{32.5 (g-4.)}{g}$. Then we only need to solve 
$$
 45.5\, +10.5625 g+\frac{169.}{g}\geq 133.2.
$$
The solution to the above inequality is $0\leq g \leq 3.04027$(this case is obviously impossible) or $g\geq 5.26269$. Hence the minimum rate of growth is 5.26 pounds a day.


\end{document}